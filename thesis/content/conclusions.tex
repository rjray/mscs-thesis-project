\section{Conclusions}
\label{sec:conclusions}

In choosing performance, expressiveness and energy use as the three metrics to measure on, the goal was to provide an environment in which any of the five languages would have the chance to stand out. The results of this research can be considered successful: each of the five languages managed to be at least as high as the \nth{2} position in at least one metrics table. The methodology itself was demonstrated to evaluate disparate languages in a fair manner, even when some measurements were significantly disproportionate.

In the changing landscape of priorities, where security and power usage become as important as performance, Rust can be recommended for this field of computing not only for its approach to memory safety, but also on the merits of raw performance and lower energy consumption. The methodology developed and demonstrated here showed a difference of nearly 14\% between Rust and the next-lowest energy usage score. That level of difference in energy efficiency has real-world implications that cannot be ignored.

Additionally, the novel DFA-Gap algorithm was shown to be effective at approximate-matching while also being simple to implement. When directly compared to the use of an existing regular expression engine, it consistently performed better for the non-interpreted languages.

The DFA-Gap algorithm has shown that it can be applied in cases where an edit distance-based algorithm yields unsatisfactory results. Sequence alignments can be computed with more control over the gaps between nucleotides. With the distinction of its approach to defining and constraining gaps, it can be further developed as an additional tool for researchers to use. Future work can also include greater analysis of and experimentation with the novel algorithm; based on the structure of Aho-Corasick it is very possible that the algorithm can be extended to multiple-pattern matching, as well.

Presented with a larger dataset, how would these metrics change? A greater selection of source code files could lead to more precision in the expressiveness measurements, while more data points would have a similar effect on the scores for energy usage and performance. Where memory safety was only briefly addressed here, additional work and research could add consideration of memory management and memory issues to the metrics. This could take the form of an additional facet of expressiveness, or even become its own metric. It could be possible to refine the methodology in ways that weigh the different metrics as opposed to treating them equally. It might then be further refined in new ways that would allow someone to choose the weights of the various metrics based on their specific needs and goals.

In the end, where energy efficiency is as important as performance, this methodology has shown its ability to clearly evaluate the suitability of a programming language.
