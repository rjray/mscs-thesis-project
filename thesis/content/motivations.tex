\section{Motivations and Prior Work}
\label{sec:motivations}

This section will discuss some of the motivations that lead to researching this specific topic. Additionally, a range of prior works will be discussed.

\subsection{Programming Languages}

Programming languages as a concept and field has always been of strong interest. Having started out with the BASIC language on a personal computer, the introduction of a second programming language (Pascal in this case) showed that there were many other languages and that their syntax and structure could vary wildly. Pascal led to C, with academic diversions into Lisp and Fortran during undergraduate study.

Working in the software industry brought about experience with a long list of other languages. These languages ranged from interpreted scripting languages (Perl, Python, Tcl, Ruby), to web-development (PHP, JavaScript), to Java and languages built on the Java Virtual Machine (Clojure, Scala). Along the way, one constant remained: an interest in comparing languages not only on speed but also on readability, expressiveness, and capability.

A language must be able to perform, but it must also be understandable. Jokes about the relative readability and maintainability of different languages date back to the APL language if not earlier than that. Setting aside endeavors such obfuscated code competitions, some languages are simply harder to read than some of their peers. Perl and Python are often compared, for example. Perl's syntax relies heavily on use of non-alphabetic symbols in using and referencing variables. Contrast this with Python's syntax, which is closer to that of C and other similar languages.

A full treatment on the basis of programming language design is outside the scope of this writing. Instead, issues of the more aesthetics-oriented language differences will be addressed by examining some static aspects of code, aspects that are completely independent of the running of the programs themselves.

\subsection{Performance, Expressiveness, Energy}

As given in section~\ref{subsec:comparison}, the experiments that will be described in this paper are designed to focus on a trio of aspects of concern to modern software developers: how well the code performs, how easy it is to read and maintain the code, and (more recently) how the code ranks in terms of energy efficiency.

The overall performance of programs is an issue often discussed when languages are compared directly. Languages such as C and C++ offer high performance, while interpreted languages like Perl and Python have comparatively poor performance. And yet, Python holds great popularity in many sub-fields such as data science, machine learning, data visualization, and task automation. It leads to the question: why would a language so much slower would be so popular?

While there are many varied reasons why people developing software like or dislike a given language, certain aspects often rise up in conversations. These aspects include the \textit{friendliness} of the language, the \textit{ease of use} it offers, and the \textit{readability} of the language. Aspects like this are sometimes referred to as the \textit{expressiveness} of a language: the breadth of ideas that can be represented in that language, and the degree to which they can be understood and communicated.

Expressiveness can be a significant factor in language selection and use. In~\cite{berkholz}, Berholz looks at measuring expressiveness by looking at how many lines of code change in an average version control commit for projects written in a range of languages. He found functional languages such as Lisp and Haskell to be the most expressive, and domain-specific languages to be biased towards high levels of expressiveness.

In addition to performance and expressiveness, the energy usage of software is rapidly becoming more important as data-centers try to reduce carbon footprints and developers target battery-driven devices like mobile phones. In~\cite{pereira}, Pereira et al do an extensive analysis of energy efficiency at the programming language level. Upon seeing the results and the methods used, it became clear that this could be used with the previous two metrics to evaluate a set of languages in even broader terms.

\subsection{The Motivations}

This research began initially as an effort to demonstrate the suitability of the Rust programming language for the bioinformatics field of computer programming. An exploration of the Rust-Bio project~\cite{rust} led to finding the SeqAn\footnote{SeqAn: \texttt{https://www.seqan.de/}} project for C++. Further investigation led to an understanding of the ongoing popularity of languages like Perl and Python in this area. It was then decided that, rather than focus specifically on Rust and its potential, this effort would instead pursue an understanding of the relative power of the selected set of languages on the metrics described above.

It is believed that these three measurements can evaluate the languages with enough clarity that a programmer could then make an informed choice as to which would better meet their needs.

\subsection{Prior Work}

Some of the papers that informed this research include:

Rahate and Chandak~\cite{rahate} performed a study very similar in nature and scope as what is planned here. From this paper, it was determined that there should be at least five algorithms under consideration and that algorithms such as Knuth-Morris-Pratt~\cite{knuth} would make good candidates.

In~\cite{chen2021string}, Chen and Nguyen describe an approach to string matching over DNA data with $k$ differences. Their technique was based primarily on \textit{edit distance}, and lent ideas to the development of the $k$-gap approximate-matching algorithm that will be described in~\ref{subsec:dfa_gap}.

Neamatollahi et al~\cite{neamatollahi} describe three pattern matching algorithms that are specifically targeted at searches on large DNA sequences. While the first is a more traditional character-based matching algorithm, the second and third take advantage of aspects of the CPU such as word-width to speed up comparisons.

The concept of multiple-pattern-matching for exact matches is explored by Bhuka and Somayajulu in~\cite{bhukya}. Their approach is based on the use of pair indexing for both the sequence and the pattern. This paper gave weight to the concept of multi-pattern matching and led to the decision to take the Aho \& Corasick algorithm~\cite{aho} as one of the evaluation algorithms.

In~\cite{cheng2003approximate}, Cheng et al describe a novel data structure and use it in two new parallel approximate matching algorithms. Ultimately, this was not used directly as the multiple-machine clusters would have greatly increased the complexity of taking energy usage measurements.
